\nonstopmode{}
\documentclass[a4paper]{book}
\usepackage[times,inconsolata,hyper]{Rd}
\usepackage{makeidx}
\usepackage[utf8]{inputenc} % @SET ENCODING@
% \usepackage{graphicx} % @USE GRAPHICX@
\makeindex{}
\begin{document}
\chapter*{}
\begin{center}
{\textbf{\huge Package `EpigeneticAgePipeline'}}
\par\bigskip{\large \today}
\end{center}
\inputencoding{utf8}
\ifthenelse{\boolean{Rd@use@hyper}}{\hypersetup{pdftitle = {EpigeneticAgePipeline: What the Package Does (Title Case)}}}{}
\begin{description}
\raggedright{}
\item[Type]\AsIs{Package}
\item[Title]\AsIs{What the Package Does (Title Case)}
\item[Version]\AsIs{0.1.0}
\item[Author]\AsIs{Who wrote it}
\item[Maintainer]\AsIs{The package maintainer }\email{yourself@somewhere.net}\AsIs{}
\item[Description]\AsIs{More about what it does (maybe more than one line)
Use four spaces when indenting paragraphs within the Description.}
\item[License]\AsIs{GPL-3}
\item[Encoding]\AsIs{UTF-8}
\item[Imports]\AsIs{BiocManager, minfi,
IlluminaHumanMethylationEPICanno.ilm10b4.hg19,
IlluminaHumanMethylationEPICmanifest,
IlluminaHumanMethylation450kmanifest,
IlluminaHumanMethylation450kanno.ilmn12.hg19, methylclock,
DunedinPACE, FlowSorted.CordBlood.450k, tidyverse, ggplot2,
ggpubr, umap, lme4, dnaMethyAge, glmmTMB, reshape2}
\item[NeedsCompilation]\AsIs{no}
\end{description}
\Rdcontents{\R{} topics documented:}
\inputencoding{utf8}
\HeaderA{main}{Documentation Epigenetic Age Pipeline}{main}
%
\begin{Description}
The 'Epigenetic Age Pipeline' is a comprehensive function designed for processing and analyzing DNA methylation data. It performs data normalization and analysis using specified parameters and input files.
\end{Description}
%
\begin{Usage}
\begin{verbatim}
  main(directory = getwd(), normalize = TRUE, useBeta = FALSE, arrayType = "450K")
\end{verbatim}
\end{Usage}
%
\begin{Arguments}
\begin{ldescription}
\item[\code{directory}] 
Directory containing input data files (default: current working directory).


\item[\code{normalize}] 
Logical. Perform data normalization if 'TRUE'.


\item[\code{useBeta}] 
Logical. If 'TRUE', expect input data as beta values (scaled between 0 and 1). If 'FALSE', process raw intensity data.


\item[\code{arrayType}] 
Type of DNA methylation array used (options: "27K", "450K", or "EPIC").

\end{ldescription}
\end{Arguments}
%
\begin{Details}
Input Files:

'Sample\_Sheet.csv':
.csv file containing phenotypic data for each sample

'IDAT Files':
IDAT files containing methylation data for each sample

'betaValues.csv':
If IDAT files are not available, processed beta values can be provided

Naming Conventions for Data in "Sample\_Sheet.csv":

'Array':
Array plate name in format "RXCY" (X: row number, Y: column number).
"Array" column should be in "RXCY" format for row and column data extraction.

'Age':
Age values of samples.

'Sex':
Gender information of samples (M or F, 0 or 1)

'Smoking\_Status':
Smoking status of samples.

'Batch':
Batch information of samples.

'Slide':
Slide information of samples.

'Bcell', 'CD4T', 'CD8T', 'Gran', 'Mono', 'nRBC':

Optional: cell type proportions if IDAT files are not available
\end{Details}
%
\begin{Value}
The function returns results and analysis plots.
\end{Value}
\printindex{}
\end{document}
